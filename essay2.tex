\documentclass{article}
\usepackage{geometry}
\usepackage{nopageno}
\geometry{top=1.0in, bottom=1.0in, left=1.0in, right=1.0in}

\newcommand{\ie}{{\em i.e.,}~}
\newcommand{\eg}{{\em e.g.,}~}

\begin{document}
	
\pagestyle{plain}

\textbf{Dr. Anita Borg was a visionary and leader among women in technology. Please give us 2-3 examples of how you have exhibited leadership. Explain how you were influential and what you were trying to achieve. These need not be demonstrated through formal or traditional leadership roles. Think broadly and examine the many ways you are having an effect on the members of your technical community, your university, or your broader community.}\\

% links:  mentoring women, keeping things organized
During my time at UC Berkeley, I have been involved with several leadership activities.
In each, I have applied my organizational skills to mentor and encourage graduate and undergraduate students in the department, especially women.

% twice as long as it should be!  Want 150-200 words per activity plus some intro & conclusion
For the past year and a half, I have been supervising an undergraduate research assistant named Durga.
When I did undergraduate research, I really admired how my graduate mentor taught me about new concepts, encouraged me to grow professionally, and believed that I would succeed.
I seek to emulate her approach as I mentor undergraduates.
First, I taught Durga about a variety of tools that I use for my research, from version control and dependency management to new programming languages and frameworks.
I also helped her learn about how to present her work.
We submitted an extended abstract and poster to CSCW 2012, and I took the opportunity to emphasize the value of identifying the most important contributions of the work and getting feedback on her writing to ensure that the story is clear.
Her submission was accepted, and she will present the work at CSCW this year.
It was most rewarding when Durga shared that because of this experience, she is confident that she can succeed in graduate school.
Mentoring has benefitted me as well; I have grown as a manager, and I have been able to do more research than I would have on my own.
%Also, when another undergraduate student, Lauren, recently asked my advisor for a research opportunity, he sent her to me.

My management skills also helped when I served as a teaching assistant (TA) to one of my advisors, Armando Fox, for a software engineering course.
Along with the other graduate TA, and with input from our advisor, I managed the undergraduate TAs, held discussion section and review sessions, and designed and graded assignments and exams.
The course focused on building a Ruby on Rails web application as a group, and I supervised half of the teams.
This involved assessing their progress, suggesting new features, and helping them work out interpersonal conflicts.
I helped my advisor run an organized class so the students would not only gain technical skills but also learn how to deal with their future collaborators.
It was great to see one of the students I supervised get hired at PivotalLabs, putting into practice what she had learned in our class.
I also received encouraging feedback on my evaluations; many students said that I was helpful and approachable, and one said ``she rocks!"

My involvement with our department's group for female graduate students, Women in Computer Science and Engineering (WICSE), has been another great leadership activity.
WICSE provides opportunities to mentor younger graduate students and undergraduates.
As part of our Big Sisters/Little Sisters program, I meet up with female first-year graduate students and help them get oriented to life in Berkeley.
I also participate in the monthly mentoring lunch, which involves helping undergraduates choose classes and get involved with research.
In addition, WICSE provides a weekly lunch for female graduate students.
In the past, the lunch was fairly disorganized; people ordering lunch were often confused about what to bring and how to get reimbursed, and the location would often be changed at the last minute.
I therefore volunteered to be lunch coordinator, and I held this position for two years.
My goal was to maintain the camaraderie by taking care of the details, so I centralized the supplies, arranged room reservations, recruited volunteers, and scheduled speakers.
My successor continued implementing things according to the pattern I had established.

My leadership activities have been an important element of my time at UC Berkeley.
I feel that I have been able to apply my skills to help others while learning a lot myself.

\end{document}